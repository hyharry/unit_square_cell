%\documentclass[10pt,a4paper]{article}
\documentclass[10pt,a4paper]{scrartcl}
%\documentclass[10pt,a4paper]{scrreprt}
\usepackage[utf8]{inputenc}
\usepackage{amsmath}
\usepackage{amsfonts}
\usepackage{amssymb}
\usepackage{graphicx}
\usepackage[left=2cm,right=2cm,top=2cm,bottom=2cm]{geometry}
\author{Yi Hu}
\title{Homogenization for Multi Field Modelling}
\subtitle{Part I: Theories and FEniCS}

\begin{document}

\begin{abstract}
In this student project, computational homogenization is utilized to investigate behaviour of composites under multiple fields. Many applications can be found in the field of material modelling, specifically for coupled problems of composites, e.g. electroactive polymers. As for implementation, a novel Finite Element framework, FEniCS, is investigated. FEniCS is a collection of libraries and modules, which uses Python (or C++) as its interface language. The most important feature about FEniCS is its specialization in code generation with respect to bilinear forms, linear forms and function spaces in the formulation of computational problems, which can translate the mathematical language directly into codes and accelerates the trial process of new models and new computation methods. Moreover, Python make the implementation of a problem even faster, as it is a dynamical language for prototyping. In the framework of FEniCS a unit cell module is realized, where the calculation including homogenized properties of composites in micro scale under multiple fields could be performed.

\end{abstract}


\end{document}

