%\documentclass[10pt,a4paper]{article}
\documentclass[10pt,a4paper]{scrreprt}
\usepackage[utf8]{inputenc}
\usepackage{amsmath}
\usepackage{amsfonts}
\usepackage{amssymb}
\usepackage{graphicx}
\usepackage[left=2cm,right=2cm,top=2cm,bottom=2cm]{geometry}
\author{Yi Hu}
\title{Homogenization for Multi Field Modelling}
\subtitle{Part I: Theories and FEniCS}

\begin{document}

\chapter{Introduction}

Composites play an important role in engineering, as they combine the advantages of each components in materials. In order to model composite materials, different materials and different material configurations need to be considered. There are several methods to achieve this modelling. Intuitively a full simulation can be realized, where micro structures and different materials would be represented explicitly in the simulated object. This results in a tremendous many degrees of freedom consuming not only large amount of memory but also the computation resources. Hence a full simulation accounting for all the micro structures and inclusions is not the most efficient way.\\

A multi scale model is conducive in modelling [search], where the deformation in small scale will be reflected in the corresponding macro scale. Moreover scale separation would benefit from various computing techniques such as parrellelization and model reduction[ref]. These techniques would possibly boost the efficiency of computation. \\

There are several approaches tackling the multi scale problem [search and ref].

In this work Homogenization Method is used.


\end{document}
