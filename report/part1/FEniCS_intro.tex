%\documentclass[10pt,a4paper]{article}
\documentclass[10pt,a4paper]{scrreprt}
\usepackage[utf8]{inputenc}
\usepackage{amsmath}
\usepackage{amsfonts}
\usepackage{amssymb}
\usepackage{graphicx}
\usepackage[left=2cm,right=2cm,top=2cm,bottom=2cm]{geometry}
\usepackage{natbib}
\usepackage{bm}

\usepackage{pythonhighlight}

% integral d
\newcommand{\myd}{\;\mathrm{d}}
% overbar
\newcommand{\overbar}[1]{\mkern 1.5mu\overline{\mkern-1.5mu#1\mkern-1.5mu}\mkern 1.5mu}

\author{Yi Hu}
\title{Homogenization for Multi Field Modelling}
\subtitle{Part I: Theories and FEniCS}

\begin{document}

\chapter{Finite Element Framework FEniCS}

The FEniCS project was started at University of Chicago and Chalmers University of Technology in 2003. The name ``FEniCS" consists ``FE", ``Finite Element" and ``CS", ``Computational Software". ``ni" makes the name sound like ``phenix". It is a collection of packages that make the best of each package to realize mathematical modelling. The main interface language of FEniCS is Python, which is a fast language for prototyping, while the most of code and libraries are implemented in C++, whose efficiency in computing stands out. Various linear algebra backends and its parallelization tools improve its efficiency further. As for modelling, UFL (Unified Form Language) and FFC (FEniCS Form Compiler) provide direct translation of mathematical formulation such as linear form and bilinear form into symbolic code. It can be regarded as the language of modelling, which simplifies the modelling process to a large extent \citep{kirby2006compiler}.

This part concentrates on the introduction of FEniCS. Firstly the structure and components of FEniCS are presented. Then a basic example is given in order to clarify its efficiency in modelling. At last some key features of FEniCS are discussed, which demonstrates its flexibility in simulations.

%--------------------------------
\section{The Structure of FEniCS}
The structure of FEniCS is presented in the following diagram. It is shown that the structure is split into three major parts. Each part represents one or several functionalities. For form compiler UFL language is involved. With the help of form compiler (FFC), C++ code of the whole model is obtained. Till this point the mathematical derivation, which is the weak form of a PDE in most cases, is converted into pure symbolic efficient C++ code. The interface between symbolic code and numerical code is called UFC (Unified Form-assembly Code). It works as a fixed interface that achieve efficient assembling of finite elements \citep{wells2012automated}.

\begin{figure}[h]
\center
\label{fig: fenics blocks}
\includegraphics[width=0.6\linewidth]{../pics/fenics_building_blocks.png}
\caption{Building Blocks of FEniCS \citep{wells2012automated}}
\end{figure} 

Numerical algorithms enter after the UFC interface. The main package that carries the numerical algorithm is DOLFIN. DOLFIN's task is to wrap functionalities of each components and their interactions. The user interface of DOLFIN is Python.  

Other building blocks such as mshr for geometrical modelling and mesh generation, FIAT (FInite element Automatic Tabulator) for automatic finite element generation for arbitrary orders, and linear algebra backends work as add-ons for FEniCS, which provides the possibility of extension. The visualization package in use is VTK, a substitute of the old one, Viper. The output files could be in various formats. Therefore software like ParaView can be applied in terms of visualization. The full structure of FEniCS can be viewed in the following diagram.

\begin{figure}[h]
\center
\label{fig: fenics map}
\includegraphics[width=0.5\linewidth]{../pics/fenics_map.png}
\caption{Components of FEniCS \citep{wells2012automated}}
\end{figure} 

%--------------------------------
\section{Basic Poisson Problem}
An implementation of a basic Poisson problem is presented in the official tutorial of FEniCS. This example is reproduced here to give a quick overview of its usage and present its simplicity in modelling \citep{wells2012automated}.

\begin{python}
from dolfin import *

# Create mesh and define function space
mesh = UnitSquareMesh(6, 4)
V = FunctionSpace(mesh, 'Lagrange', 1)

# Define boundary conditions
# Value function for boundary condition x[0]->x, x[1]->y
u0 = Expression('1 + x[0]*x[0] + 2*x[1]*x[1]')
# Mark boundary
def u0_boundary(x, on_boundary):
    return on_boundary
# Add Dirichlet boundary condition
bc = DirichletBC(V, u0, u0_boundary)

# Define variational problem
u = TrialFunction(V)
v = TestFunction(V)
f = Constant(-6.0)
# Bilinear Form
a = inner(nabla_grad(u), nabla_grad(v))*dx
# Linear Form
L = f*v*dx

# Compute solution
u = Function(V)
solve(a == L, u, bc)

# Plot solution
plot(u, interactive = True)

# Dump solution to file in VTK format
file = File('poisson.pvd')
file << u
\end{python}

The mathematical model of this problem is 
\begin{equation}
\label{eq: poisson}
\left\{
\begin{array}{ll}
-\Delta u(\mathbf{x})=0 & \text{in} \ \Omega \\
u(\mathbf{x})=u_{0} & \text{on} \ \partial \Omega .
\end{array}
\right.
\end{equation}
The corresponding weak form can be written as
\begin{equation}
\label{eq: weak}
a(u,v) = L(v), \ \forall v \in \hat{V}
\end{equation}
with the bilinear form and linear form defined as
\[
a(u,v) = \int_{\Omega} \nabla{u} \cdot \nabla{v} \myd{\mathbf{x}}, \ L(v) = \int_{\Omega} fv \myd{\mathbf{x}}.
\]
The sought after function is $u$ in function space $V = \left\lbrace v \in H^{1}\left( \Omega \right) : v=u_{0} \  \text{on} \ \partial \Omega \right\rbrace$ and the according $\hat{V}$ is defined as $\hat{V} = \left\lbrace v \in H^{1}\left( \Omega \right) : v=0 \  \text{on} \ \partial \Omega \right\rbrace$. 

As can be immediately seen in the code, every steps in finite element analysis are well preserved in FEniCS. Moreover, the name of functions and classes are consistent with mathematical language. The modelling always begins with defining its geometry and function spaces. This setting is similar with solving a PDE in weak form, where its domain and function spaces are usually given in condition. Geometry and mesh can also be imported using external files using various formats, such as \texttt{.msh} generated by \texttt{gmsh}. One thing to notice is that element type should be defined when defining a function space. As shown in the example \texttt{'Lagrange’}. The order of element is simply 1.

It follows with defining the boundary condition such as Dirichlet boundary condition. An \texttt{Expression} is a class holding the code of creating a mathematical function in the domain. It is extensible with C++ code. Here we use it to define the boundary value. As for imposing this boundary condition, one need to mark the boundary first. Marking other boundary entities such as points, lines, and facets is the same with the provided code structure. \texttt{on\_boundary} is only a predefined marker that will return \texttt{True} when on the boundary.

Then the mathematical model is conveniently expressed with \texttt{a\ =\ inner(nabla\_grad(u),\ nabla\_grad(v))*dx} and \texttt{L\ =\ f*v*dx}, which is rather straightforward.

Next is the solving step, where one should redefine a new \texttt{Function} to hold the solution. Here overriding the above defined \texttt{u} is implemented. Post processing such as visualization of the result follows with \texttt{plot}. Output of the result is also standard with the listed commands.

The presentation of the basic Poisson problem serves as an introduction to the functionalities of FEniCS. To sum up, the implementation is in accordance with mathematical notations. Pre- and post-processing of the computation can be realized using handy functions supported by FEniCS. More useful features are discussed in the following section.

%--------------------------------
\section{Relevant Key Features}
When the problem becomes more complicated, there are more advanced and handy tools coming into play. Here we list only a few that appear in the material modelling and multi field problems.

First we explore the function space definition in FEniCS. There are several kinds of \texttt{FunctionSpace} in FEniCS. If a vector or tensor is used, \texttt{VectorFunctionSpace} or \texttt{TensorFunctionSpace} should be defined. More powerful structure is the \texttt{MixedFunctionSpace}, which receives a list of \texttt{FunctionSpace} and composes them together into a merged one. Due to the general representation of function spaces, arbitrary order of elements and various combinations possible. For very specific function spaces which require periodicity \texttt{constrained\_type} keyword is set to be a periodic mapping. This mapping is derived from \texttt{SubDomain}. Much caution needs to be devoted into the definition of mapping, which can be observed in the corresponding code for material modelling.

\begin{python}
# Useful properties and variables for FunctionSpace definition
FS = FunctionSpace(mesh=some_mesh, family='Lagrange', degree=2, constrained_domain=some_defined_periodic_mapping)

# Vector and tensor function spaces
VFS = VectorFunctionSpace(mesh, 'Lagrange', 1)
TFS = TensorFunctionSpace(mesh, 'Hermite', 1)

# Merged function Space
MFS = MixedFunctionSpace([FS, VFS, TFS])
\end{python}

When dealing with multi field modelling, not only merging functions but also splitting functions are required, since in the formulation different terms are calculated with different functions. A mixed function is generated with a \texttt{MixedFunctionSpace} and the corresponding component functions are obtained through \texttt{split} without losing dependency to the merged function. In the problem formulation it is often the case that inner product, outer product, derivative, and integral etc. are built. This is achieved easily with the help of UFL operators. Some of the operators such as \texttt{nabla\_grad()} and \texttt{inner()} are already mentioned above. One strength of FEniCS is that this formulation is not restricted in the vector and matrix, but applicable on tensors of higher order. The index notation representation is also valid in the formulation, namely \texttt{i,j,k,l}, obeying Einstein's summation convention. After the derivation a stiffness matrix is assembled with \texttt{assemble}. These usages are as follows.

\begin{python}
# Merged function space
MFS = MixedFunctionSpace([FS, VFS, TFS])

# Merged function and split functions
merged_f = Function(MFS)
(f, vf, tf) = split(merged_f)

# Integral
L = f*v*dx

# Derivative and its alternative
L_derivative = derivative(L, f, df)
L_diff = diff(L, f)

# Index notation
i, j, k, l = indices(4)
Energy = 0.5*F[i,j]*C_i_[i,j,k,l]*F[k,l]*dx

# Stiffness matrix, note that test function and trial function should be initiated first
f_test = TestFunction(FS)
f_trial = TrialFunction(FS)
a = f_test*C*t_trial*dx
K = assemble(a)
\end{python}

The last usage is particularly useful in composite modelling, i.e. defining different domains. Subdomains are defined through subclassing of \texttt{SubDomain}. Overriding its member method \texttt{inside()} gives the condition to distinguish whether it is in a specific subdomain or not. In integral the differentials are set according to subdomain labels.

\begin{python}
# Inclusion definition
class Inclusion(SubDomain):
    def inside(self, x, on_boundary):
        d = sqrt((x[0] - 0.5)**2 + (x[1] - 0.5)**2)
        return d<0.25 or near(d,0.25)

# Initiate an instance
circle = Inclusion()

# Set cell domains for integral
domains = CellFunction("size_t", mesh)
domains.set_all(0)
# Mark inclusion
circle.mark(domains,1)

# Integrate accordingly
dx = Measure('dx', domain=mesh, subdomain_data=domains)
Pi = psi_m*dx(0) + psi_i*dx(1)
\end{python}

\end{document}
