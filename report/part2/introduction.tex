%\documentclass[10pt,a4paper]{article}
\documentclass[10pt,a4paper]{scrreprt}
\setlength{\parindent}{0pt}
\setlength{\parskip}{1.5ex}

\usepackage[utf8]{inputenc}
\usepackage{amsmath}
\usepackage{amsfonts}
\usepackage{amssymb}
\usepackage{graphicx}
\usepackage{natbib}

\usepackage[left=2cm,right=2cm,top=2cm,bottom=2cm]{geometry}
\author{Yi Hu}
\title{Homogenization for Multi Field Modelling}
\subtitle{Part II: Implementation and Numerical Examples}

\begin{document}

\chapter{Introduction}

In the first part of this report Homogenization Theory and FEniCS framework are introduced, which constitute the basis of our simulation. The current part focus on the implementation and numerical results. A Unit Cell module is established in FEniCS framework using Python language. The most functionalities of homogenization in composite material modelling are realized in this module.

The approach in this Unit Cell module is motivated by work in the group. The micro-macro coupling part of this module originates from \citep{miehe_computational_1999-1}, where the homogenization of large strain deformation is developed. As for implementation of Homogenization Theory in FE$^{2}$ scheme, matrix representation and algorithms for small strain problem are presented in detail in \citep{miehe_computational_2002}. Various kinds of boundary conditions are also taken into account in \citep{miehe_computational_2002}. The reformulation of the homogenization problem into a optimization context is introduced in \citep{miehe_strain-driven_2002}\citep{miehe_computational_2003} \citep{miehe_homogenization_2002}, which extends the application of Homogenization in material modelling. It is a great unification of the material modelling, since various kinds of materials could fit in this scheme, e.g. plasticity material where Karush-Kuhn-Tucker condition in plasticity modelling is also a optimization procedure.

Homogenization of composite material in multiple fields can be achieved. Previous work has been done by a lot of researchers. The method used in the current work is inspired by \citep{keip_two-scale_2014} and \citep{schroder_two-scale_2012}. In the case of multiple fields modelling becomes more intricate. Energy approach will lead to simpler representation of the problem and unravel the basic characteristics of material properties. The formulation in \citep{miehe_incremental_2011} is a good example of this kind. The Legendre transformation performed in the literature provides a good feature of the energy based approach. More recent work about algorithms in magneto-electro-mechanically coupled problem can be referred to \citep{schroeder_algorithmic_2016} and \citep{miehe_homogenization_2016}. In the second literature the stability analysis of coupled problem is also discussed in fully detail, which is an important issue in multi field modelling and homogenization \citep{geers_multi-scale_2010}.

As for the implementation part FEniCS book \citep{wells2012automated} gives extensive instructions about the functionalities of FEniCS. A Unit Cell module is implemented on this basis. To test the module several numerical examples are simulated. These examples starts from the basic hyperelastic composite material with circle inclusions to the three dimensional coupled field problem. 

This part of the report is arranged in the following order. First we present the main method especially for the coupled field modelling. This is followed by the ideas behind implementation and introduction about module and its functionalities. Then a section about numerical examples is provided. In the end of this part a summary concludes all the work that has been done in this ``Forschungsmodul" and the ongoing improvement in the future.

\end{document}
