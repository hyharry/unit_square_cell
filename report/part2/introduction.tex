%\documentclass[10pt,a4paper]{article}
\documentclass[10pt,a4paper]{scrreprt}
\setlength{\parindent}{0pt}
\setlength{\parskip}{1.5ex}

\usepackage[utf8]{inputenc}
\usepackage{amsmath}
\usepackage{amsfonts}
\usepackage{amssymb}
\usepackage{graphicx}
\usepackage{natbib}

\usepackage[left=2cm,right=2cm,top=2cm,bottom=2cm]{geometry}
\author{Yi Hu}
\title{Homogenization for Multi Field Modelling}
\subtitle{Part II: Implementation and Numerical Examples}

\begin{document}

\chapter{Introduction}

The approach in this unit cell module is motivated by the work in the group. The micro-macro coupling part of this module originates from \citep{miehe_computational_1999-1}, where the homogenization of large strain deformation is developed. As for the implementation of computational homogenization in FE$^{2}$ scheme, matrix representation and algorithms for small strain problems are presented in detail in \citep{miehe_computational_2002}. Various kinds of boundary condition are also taken into account in \citep{miehe_computational_2002}. The reformulation of the homogenization problem into a optimization context is introduced in \citep{miehe_strain-driven_2002}\citep{miehe_computational_2003} \citep{miehe_homogenization_2002}, which extends the application of computational homogenization in material modelling. It is a great unification of the material modelling, since various kinds of materials could fit in this scheme, e.g. plasticity material where Karush-Kuhn-Tucker condition in plasticity modelling is also a optimization procedure.

Homogenization of composite material in multiple fields can be achieved. Previous work has been done by a lot of researchers. The method used in the current work is inspired by \citep{keip_two-scale_2014} and \citep{schroder_two-scale_2012}. In the case of multiple fields modelling the formulation becomes more intricate. An energy approach will lead to a simpler representation of the problem and unravel the basic characteristics of material properties. The formulation in \citep{miehe_incremental_2011} is a good example of this kind. The Legendre transformation performed in the literature provides a good feature of the energy based approach. For more recent work about algorithms in the magneto-electro-mechanically coupled problem we refer to \citep{schroeder_algorithmic_2016} and \citep{miehe_homogenization_2016}. The latter also discusses the stability analysis of coupled problem in fully detail, which is an important issue in multi field modelling and homogenization \citep{geers_multi-scale_2010}.

As for the implementation part the FEniCS book \citep{wells2012automated} gives extensive instructions about the functionalities of FEniCS. A unit cell module is implemented on this basis. To test the module several numerical examples are simulated. These examples start from a basic hyperelastic composite material with circle inclusions to three dimensional coupled field problems. 

This part of the report is arranged in the following order. First we developed the computational homogenization especially for coupled field modelling. This is followed by the ideas behind implementation and introduction about module and its functionalities. Then a section about numerical examples is provided. The report closes with a summary concluding all the work that has been done in this ``Forschungsmodul" and the ongoing improvement in the future.

\end{document}
