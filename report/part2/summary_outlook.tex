%\documentclass[10pt,a4paper]{article}
\documentclass[10pt,a4paper]{scrreprt}
\usepackage[utf8]{inputenc}
\usepackage{amsmath}
\usepackage{amsfonts}
\usepackage{amssymb}
\usepackage{graphicx}
\usepackage[left=2cm,right=2cm,top=2cm,bottom=2cm]{geometry}

\usepackage{bm}

\usepackage{pythonhighlight}

% integral d
\newcommand{\myd}{\;\mathrm{d}}
% overbar
\newcommand{\overbar}[1]{\mkern 1.5mu\overline{\mkern-1.5mu#1\mkern-1.5mu}\mkern 1.5mu}

\author{Yi Hu}
\title{Homogenization for Multi Field Modelling with FEniCS}
\subtitle{Part II: Implementation and Numerical Examples}

\begin{document}

\chapter{Summary and Outlook}
This ``Forschungsmodul" concentrates on homogenization method for multi field modelling, where the novel finite element framework FEniCS is investigated. Some key points of homogenization and basic concepts in FEniCS are presented in this report. A unified derivation of multi-field problem is given in the previous chapters. The strength of FEniCS is revealed for problems of multiple fields. It is also demonstrated that the modelling process in FEniCS is rather straightforward, as it embeds mathematical notations in its framework. The time of prototyping is further shortened, which is  benefited from the features of Python.

It is also shown that the current module is easy to extend for other problems, as the most classes are rather generic in the formulation. This will lead to difficulty, if one is not familiar with the methods and notations in the original code. In order to alleviate this drawback, docstrings explaining the code are well preserved and an example manual will guide users to better understand the usage of the code.

As for improvement of the code, the most relevant one is to expand the functionalities of this unit cell module, which adapts to other numerical scheme. One of such is $\text{FE}^{2}$. Much work can be done in this direction in implementation. The upper architecture of the problem is a macro scale problem, which will use the parameters calculated from the micro scale, i.e. the homogenized problem. Since the parameters vary from element to element, the assembling might be a problem. If one wants to make the best of the efficiency of FEniCS, this parameter data should be transformed into a Function object in FEniCS (or Expression object, it would be more efficient, if subclassing of Expression is written in C++). This will add an extra layer of data communication. If this subclassing is achieved and linked to the unit cell module and succeed in handling parallelization nature of FEniCS. It would become a powerful tool to investigate composite behaviour in multiple fields. Another consideration is to design a new assembler that is appropriate for macro scale problem. The new assembler is based on the old one. One must then delve into the detailed implementation written in C++. This method is not generic, as for other type of PDEs with even more than one homogenized parameters this method can not be applied. Another improvement of the current work might be plasticity or viscosity material model. For plasticity problem, an optimization problem could be formulated in micro scale and accomplish the calculation. For viscosity the time dependency needs to be included in the implementation. Additionally for more detailed simulation the interface between different materials should be accounted.

\end{document}

