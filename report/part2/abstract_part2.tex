%\documentclass[10pt,a4paper]{article}
\documentclass[10pt,a4paper]{scrartcl}
%\documentclass[10pt,a4paper]{scrreprt}
\usepackage[utf8]{inputenc}
\usepackage{amsmath}
\usepackage{amsfonts}
\usepackage{amssymb}
\usepackage{graphicx}
\usepackage[left=2cm,right=2cm,top=2cm,bottom=2cm]{geometry}
\author{Yi Hu}
\title{Homogenization for Multi Field Modelling}
\subtitle{Part I: Theories and FEniCS}

\begin{document}

\begin{abstract}
In the first part of this report the underlying theory and the FEM framework are introduced, which constitute the basis of our simulation. The current part focus on the implementation and numerical results. A unit cell module is established in FEniCS framework using Python language. A unit cell is a Representative Volume Element (RVE) with its edge length as one. We use this term in order to be consistent with implementation. In the current implementation the unit cell is a unit square in two dimensional case and a unit cube for three dimensional. The most functionalities of homogenization in composite material modelling are realized in this module.

\end{abstract}


\end{document}

